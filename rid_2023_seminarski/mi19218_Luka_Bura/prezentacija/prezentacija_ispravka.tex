\documentclass[aspectratio=1610,17pt,utf8]{beamer}

\usepackage[utf8]{inputenc}
\usepackage[T1]{fontenc}
\usepackage[USenglish]{babel}
\usepackage{graphicx} % graphics
\usepackage{mathabx}
\usepackage{mathpazo}
\usepackage{eulervm}




% title slide definition
\title[Shorter Title]{Da li znanje treba da bude besplatno?}
\author[Shorter Author]{Luka Bura}
\institute[Matematicki fakultet]
{
	
{Prezentacija u okviru kursa:\\Računarstvo i društvo\\ Matematički fakultet\\profesor: Sana Stojanović Đurđević}
}


%--------------------------------------------------------------------
%                            Titlepage
%--------------------------------------------------------------------

\begin{document}
	
	\begin{frame}[plain]
		\titlepage
	\end{frame}
	
	
	%-------------------------------------------------------------------
	%                            Content
	%-------------------------------------------------------------------
	
	\section{Uvod}
	

	
	\begin{frame}{Uvod:}
		
		\begin{itemize}
			\item Kako je sve počelo?
			\item Aleksandra iz Kazahstana i ilegalni sajt Sci-Hub.
                \item Prednosti i mane.
                \item Sa razvojem veštačke inteligencije, interneta i novih tehnologije, pristup informacijama postaje veći i lakši.
		\end{itemize}
	\end{frame}
	
	%%%%%%%%%%%%%%%%%%%%%%

 	\section{Sci-Hub}
	
	\begin{frame}{Sci-Hub:}
		
		\begin{itemize}
			\item Sci-Hub, osnovan 2011. godine od strane kazahstanske programerke i istraživačice Aleksandre Elbakjan, predstavlja važan primer besplatnog znanja u praksi.
			\item Medutim, Sci-Hub takode suočava se sa kritikama koje se tiču kršenja autorskih prava i potencijalnog negativnog uticaja na naučne izdavače i autore.

		\end{itemize}
	\end{frame}


  	\section{Istorijat}
	
	\begin{frame}{Istorijat:}
		
		\begin{itemize}
			\item U antičkoj Grčkoj, na primer, znanje je često bilo
                široko dostupno kroz javne debate, dijaloge i biblioteke, kao što je čuvena Biblioteka u Aleksandriji.
			\item U novijoj istoriji, razvoj besplatnog znanja je bio          usko povezan sa tehnološkim inovacijama, kao što su             kompjuteri, internet i digitalni mediji.

		\end{itemize}
	\end{frame}

  	\section{Pravna strana}
	
	\begin{frame}{Pravna strana:}
		
		\begin{itemize}
  	\begin{itemize}
			\item U većini zemalja, Sci-Hub se smatra ilegalnim jer krši autorska prava izdavača koji drže licencu na te radove. Neki od najvećih izdavača naučnih radova, kao što su Elsevier, Springer Nature i Wiley, podneli su tužbe protiv Sci-Hub-a zbog povrede autorskih prava, a sajt je izgubio nekoliko sudskih sporova.
			\item Međutim, postoji debata o etici i pristupačnosti naučnih radova. Pristalice Sci-Hub-a tvrde da znanje treba biti dostupno svima bez obzira na njihovu finansijsku situaciju, dok protivnici navode da autorska prava treba poštovati.
                \item Uprkos pravnim problemima, Sci-Hub i dalje funkcioniše, koristeći različite domene i metode za održavanje svoje usluge dostupnom korisnicima širom sveta.

		\end{itemize}
		\end{itemize}
	\end{frame}
 
	\section{Prednosti}
	
	%%%%%%%%%%%
	
	\begin{frame}{Prednosti:}
		
		\begin{itemize}
			\item Jednakost
			\begin{itemize}
				\item Svako bi trebao imati pristup znanju bez obzira na svoj socio-ekonomski status.
				\item Zašto bi delili ljude prema kolicini novca koje imaju ili koje žele da odvoje za obrazovanje i nauku.
                \item Jednakost pristupa znanju takode može podstaći raznolikost u naučni istraživanjima i inovacijama.
                
			\end{itemize}
		\end{itemize}
	\end{frame}
	
	%%%%%%%%%%%
	
	\begin{frame}{Prednosti:}
		
		\begin{itemize}
			\item Inovacija:
			\begin{itemize}
				\item Slobodan pristup znanju podstiče inovacije i napredak.
				\item Slobodan pristup je jedini adekvatan način da se povezu razni ljudi iz celog sveta iz iste sfere interesovanja, i da promene svet.
                    \item Jedan od najpoznatijih primera uspeha besplatnog znanja je projekat Human Genome Project (HGP), medunarodni naučni poduhvat koji je imao za cilj da sektvenira ljudski genom. Rezultati ovog projekta su besplatno dostupni naučnicima širom sveta, što je dovelo do značajnih medicinskih otkrića, razvoja novih terapija i boljeg razumevanja ljudske genetike.
			\end{itemize}
		\end{itemize}
	\end{frame}
	
	\begin{frame}{Prednosti:}
		
		\begin{itemize}
			\item Demokratizacija:
			\begin{itemize}
				\item Kada je znanje dostupno svima, to može pomoći u stvaranju demokratskog društva.
				\item Ljudi mogu pristupiti informacijama koje su im potrebne za donošenje informiranih odluka o politici i drugim društvenim pitanjima.
				\item Znanje bi trebalo biti dostupno svima jer je to temeljni uslov da se oživi razvoj društva i planete.
                    \item Demokratizacija znanja takode može doprineti većoj raznolikosti u naučnim istraživanjima, jer omogućava istraživačima iz različitih kultura, etničkih grupa i društvenih sredina da saraduju i razmenjuju ideje.
				
			\end{itemize}
		\end{itemize}
	\end{frame}
	
	
	\begin{frame}{Prednosti:}
		
		\begin{itemize}
			\item Humanitarne svrhe:
			\begin{itemize}
				\item Slobodan pristup znanju može pomoći u rešavanju humanitarnih problema u svetu. Na primjer, znanje o medicini, poljoprivredi i tehnologiji može biti korišćeno za rešavanje problema siromaštva i bolesti u nerazvijenim zemljama.
				\item Nadarena deca mogu sama da sticu znanje i unapredjuju i sebe i drustvo.
                    \item Razne humanitarne fondacije posluju preko interneta, štiteći prava anonimnih dobrotvora. Nama najbliža je "Novak Djoković Foundation".				
				
			\end{itemize}
		\end{itemize}
	\end{frame}

         \begin{frame}{Prednosti:}
		
		\begin{itemize}
			\item Sloboda govora:
			\begin{itemize}
				\item Sloboda govora je fundamentalno pravo koje omogućava razvoj i napredak, ali i nosi izazove u pogledu etike, privatnosti i sigurnosti.
				\item Kada govorimo o besplatnom znanju i slobodi govora, moramo uzeti u obzir potrebu za transparentnošću i otvorenošću u razmeni informacija. U savremenom društvu, gde se tehnologija razvija brzo, pristup informacijama i znanju može biti ključan za rešavanje globalnih izazova, kao što su klimatske promene, socijalna nejednakost i zdravstvene krize
				
			\end{itemize}
		\end{itemize}
	\end{frame}


	\begin{frame}{Reference:}
		
		\begin{itemize}
			
			\begin{itemize}
				\item Pisani, E., AbouZahr, C. (2010). Sharing health data: Good intentions are not enough.
                    \item Bohannon, J. (2016). Who’s downloading pirated papers?         Everyone.Science
				\item Greshake, B. (2017). Looking into Pandora’s Box: The Content of Sci-Hub and its Usage
				\item Himmelstein, D. S., Romero, A. R., Levernier, J. G., Munro,T. A., McLaughlin, S. R., Greshake Tzovaras, B., Greene, C. S. (2018). Sci-Hub provides access to nearly all scholarly literature
			\end{itemize}
		\end{itemize}
	\end{frame}
    
	
	\begin{frame}{Zaključak:}
		
		\begin{itemize}
			\item Diskusija:
			\begin{itemize}
				\item Sloboda govora.
				\item Pravda.
				
				
			\end{itemize}
		\end{itemize}
	\end{frame}
	
	
	%%%%%%%%%%%
	
\end{document}
