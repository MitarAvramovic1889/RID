\documentclass[aspectratio=1610,17pt,utf8]{beamer}

\usepackage[utf8]{inputenc}
\usepackage[T1]{fontenc}
\usepackage[USenglish]{babel}
\usepackage{graphicx} % graphics
\usepackage{mathabx}
\usepackage{mathpazo}
\usepackage{eulervm}




% title slide definition
\title[Shorter Title]{Should knoweledge bee free}
\author[Shorter Author]{Luka Bura}
\institute[Matematicki fakultet]
{
	{Matematicki fakultet}
}


%--------------------------------------------------------------------
%                            Titlepage
%--------------------------------------------------------------------

\begin{document}
	
	\begin{frame}[plain]
		\titlepage
	\end{frame}
	
	\begin{frame}[plain]
		\tableofcontents
	\end{frame}
	
	%-------------------------------------------------------------------
	%                            Content
	%-------------------------------------------------------------------
	
	\section{Uvod}
	
	\begin{frame}{Uvod:}
		
		\begin{itemize}
			\item Kako je sve počelo?
			\item Aleksandra iz Kazahstana i ilegalni sajt Sci-Hub.
			\item Prednosti i mane.
		\end{itemize}
	\end{frame}
	
	%%%%%%%%%%%%%%%%%%%%%%
	
	\section{Prednosti}
	
	%%%%%%%%%%%
	
	\begin{frame}{Prednosti:}
		
		\begin{itemize}
			\item Jednakost
			\begin{itemize}
				\item Svako bi trebao imati pristup znanju bez obzira na svoj socio-ekonomski status.
				\item Zašto bi delili ljude prema kolicini novca koje imaju ili koje žele da odvoje za obrazovanje i nauku.
			\end{itemize}
		\end{itemize}
	\end{frame}
	
	%%%%%%%%%%%
	
	\begin{frame}{Prednosti:}
		
		\begin{itemize}
			\item Inovacija:
			\begin{itemize}
				\item Slobodan pristup znanju podstiče inovacije i napredak.
				\item Slobodan pristup je jedini adekvatan način da se povezu razni ljudi iz celog sveta iz iste sfere interesovanja, i da promene svet.
				\item Ukoliko se desi da ograničavamo pristup naucnim dokumentima samo prema novacnoj situaciji izgubicemo mnogo stvari koje bi doprinele bilo kojoj grani nauke.
			\end{itemize}
		\end{itemize}
	\end{frame}
	
	\begin{frame}{Prednosti:}
		
		\begin{itemize}
			\item Demokratizacija:
			\begin{itemize}
				\item Kada je znanje dostupno svima, to može pomoći u stvaranju demokratskog društva.
				\item Ljudi mogu pristupiti informacijama koje su im potrebne za donošenje informiranih odluka o politici i drugim društvenim pitanjima.
				\item Znanje bi trebalo biti dostupno svima jer je to temeljni uslov da se oživi razvoj društva i planete.
				
			\end{itemize}
		\end{itemize}
	\end{frame}
	
	
	\begin{frame}{Prednosti:}
		
		\begin{itemize}
			\item Humanitarne svrhe:
			\begin{itemize}
				\item Slobodan pristup znanju može pomoći u rešavanju humanitarnih problema u svetu. Na primjer, znanje o medicini, poljoprivredi i tehnologiji može biti korišćeno za rešavanje problema siromaštva i bolesti u nerazvijenim zemljama.
				\item Nadarena deca mogu sama da sticu znanje i unapredjuju i sebe i drustvo.
				
				
			\end{itemize}
		\end{itemize}
	\end{frame}
	
	
	\begin{frame}{Zaključak:}
		
		\begin{itemize}
			\item Diskusija:
			\begin{itemize}
				\item Sloboda govora.
				\item Pravda.
				
				
			\end{itemize}
		\end{itemize}
	\end{frame}
	
	
	%%%%%%%%%%%
	
\end{document}
